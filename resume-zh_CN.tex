% !TEX TS-program = xelatex
% !TEX encoding = UTF-8 Unicode
% !Mode:: "TeX:UTF-8"

\documentclass{resume}
\usepackage{zh_CN-Adobefonts_external} % Simplified Chinese Support using external fonts (./fonts/zh_CN-Adobe/)
% \usepackage{NotoSansSC_external}
\usepackage{NotoSerifCJKsc_external}
% \usepackage{zh_CN-Adobefonts_internal} % Simplified Chinese Support using system fonts
\usepackage{linespacing_fix} % disable extra space before next section
\usepackage{cite}
\usepackage{graphicx}
\usepackage{tabu}
\usepackage{multirow}
\usepackage{progressbar}

\begin{document}
\pagenumbering{gobble} % suppress displaying page number

\begin{center}
\Huge{个~~~人~~~简~~~历}
\ \\ \ 
\end{center}
\Large{
  \begin{tabu}{ c l l }
   \multirow{5}{1in}{\includegraphics[width=0.88in]{avatar}} &
   \scshape{李高阳} &  \\
    & 性别:男 & 民族:汉 \\
    & 电话:(+86) 15002555080 & 生日:1991-09 \\
    & 邮箱:li.gaoyang@foxmail.com & 微信:sandbox\_ligy\\
    % & 地址:甘肃省兰州市天水南路222号 \hspace{40} & %籍贯:甘肃宁县
    & 籍贯:甘肃宁县 & 政治面貌:党员
  \end{tabu}
}

\section{教育经历}
% \textbf{兰州大学}
\datedsubsection{\textbf{兰州大学硕博连读与中科院兰州近物所联合培养博士}}{2015 -- 2019}
\textit{专业:物理学\ 理论物理}\\
\textit{方向:凝聚态理论}\\
\textit{导师:罗洪刚教授(长江学者)、房铁峰教授}\\
\textit{毕业论文题目:铁磁石墨烯中近藤效应的数值重整化群研究}
\datedsubsection{\textbf{兰州大学\ 本科}}{2009 --  2013}
\textit{专业:物理学国家基地班}

\section{发表文章}
\begin{itemize}
\item \textbf{Gao-Yang Li}, Tie-Feng Fang, Ai-Min Guo, and Qing-Feng Sun \textit{Ferromagnetism-induced Kondo effect in graphene with a magnetic impurity}, Phys. Rev. B \textbf{100}, 115115 (2019).
\item Wan-Xiu He, Zhan Cao, \textbf{Gao-Yang Li}, Lin Li, Hai-Feng Lü, ZhenHua Li, and Hong-Gang Luo \textit{Performance of the T-matrix based master equation for Coulomb drag in double quantum dots}, Phys. Rev. B \textbf{101}, 035417 (2020).
\end{itemize}

\section{研究兴趣}
\begin{itemize}%[parsep=0.5ex]
\item 数值重整化群(NRG)
\item 量子多体系统的数值计算方法
\item 多体计算方法与机器学习的交叉领域
\end{itemize}

\section{其他技能}
% increase linespacing [parsep=0.5ex]
\begin{itemize}%[parsep=0.5ex]
\item 全密度矩阵数值重整化群程序(可以计算有限温度平衡态下自旋分辨的单杂质Anderson模型)
\item 熟悉常用的机器学习算法(SVM, Logistic Regression, 随机森林等),有这些算法的实际使用经验。
\item 熟悉常用的深度学习算法(ANN, CNN, LSTM, AE, DCGAN),有这些算法的实际使用经验。
\item CET-6
\end{itemize}

% \section{自我评价}
% \qquad 自认为是一个上进努力有责任心的人,能和周围的人和睦相处,进行高效的交流。自学能力强,对新知识有较高的学习热情,能够自驱地学习需要的知识,并能投入精力解决问题。

% \section{个人主页}
% \rm{https://github.com/GoldenRaven}

%% Reference
% \newpage
% \bibliographystyle{IEEETran} \bibliography{mycite}
\end{document}
